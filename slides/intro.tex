\documentclass[10pt]{beamer}

  % Math Packages
  \usepackage{amsmath}
  \usepackage{amsthm}
  \usepackage{mathtools}

  % Graphics Packages
  \usepackage{graphicx}
  \usepackage[export]{adjustbox}

  % Hyperlinks
  \usepackage{hyperref}
  \hypersetup{colorlinks=true,linkcolor=blue,urlcolor=blue}

  % Theme Settings
  \usetheme{metropolis}
  \usefonttheme{professionalfonts}

\title{Computational Economics}
\author{Chase Coleman}
\institute{Rice}
\date[]{\today}

\begin{document}


% Title Slide
\begin{frame}
  \thispagestyle{empty}
  \titlepage
\end{frame}


\section{Introduction}

  \begin{frame} \frametitle{``Goal of economic research''}
  
    Tom Sargent \href{http://www.tomsargent.com/personal/dinner_speech_2015.pdf}{described} his point
    of view of what economists do by modifying a metaphor used by Richard Feynman to describe
    astrophysics research.
  
    \vspace{0.5cm}
  
    \begin{quote}
      An astrophysicist is in the position of someone who does not know a game called chess and who
      observes an incomplete set of moves chosen by players who happen to be playing chess. From
      these, the observer's job is to discover the players' purposes and the rules of chess. For
      Feynman, the ``rules of the game of chess'' stand in for unknown laws of physics
    \end{quote}
  
  \end{frame}
  
  
  \begin{frame} \frametitle{``Goal of economic research''}
  
    Tom goes on to describe how economics differs from astrophysics
  
    \vspace{0.5cm}
  
    \begin{quote}
      Like Feynman's metaphorical physicist, our task as economics is to infer a ``game'' from
      observed data. But then we want to do something that physicists don't: to think about how
      different ``games'' might produce improved outcomes
    \end{quote}
  
  \end{frame}
  
  
  \begin{frame} \frametitle{Examples of questions in economic research}
  
    Economists have relatively broad interests and the examples below are not representative of
    everything economists might study:
  
    \begin{itemize}
      \item Imagine the world is experiencing a pandemic (not too hard for most people in this room
        to imagine what that's like\dots). What policies best manage the trade-off between the need
        to keep our economy running and the risks associated with a pandemic? See
        \href{https://www.nber.org/papers/w27102}{Acemoglu, Chernozhukov, Werning, Whinston}
      \item What are possible causes of the Black-White gap in college attainment? Are there policies
        that we could implement that would lower this gap? See
        \href{https://s3.amazonaws.com/real.stlouisfed.org/wp/2022/2022-036.pdf}{Gregory, Kozlowski, Rubinton}
      \item Governments that make commitments to do things like ``cut spending'' or ``lower
        inflation'' are often incentivized to make different choices in the future. What forms of
        commitment can help us manage these incentives and prevent future reneging on past promises?
        See \href{http://www.tomsargent.com/research/LS_tom.pdf}{Jiang, Sargent, Wang}
    \end{itemize}

  \end{frame}


\section{How/Why Computational Economics?}

  \begin{frame} \frametitle{What is ``computational economics''?}

    Computational economists attempt to build ``mathematical versions of an economy'' where they
    can perform experiments and learn about how the world might react to different policies.

    \vspace{0.5cm}

  \end{frame}


  \begin{frame} \frametitle{Why ``computational economics''?}

    \begin{itemize}
      \item \textbf{Cost}: Economic experiments can be relatively expensive -- For example,
        studies on the universal basic income are almost always restrictive in size due to costs.
        This limits the scale and generalizability of these programs. Lots of examples like this
        in other industries (car safety ratings, airplane design, etc...)
    \end{itemize}
  \end{frame}



  \begin{frame} \frametitle{Why ``computational economics''?}

    \begin{itemize}
      \item \textbf{Ethics}: It would be ethically challenging to run experiments for many
        interesting questions in economics. For example, imagine trying to measure the impact of
        a severance on how long it takes an individual to find a job. To measure this
        experimentally you might find 50,000 working individuals and fire them giving half of
        those fired a severance payment and no payment to the other half.
    \end{itemize}
  \end{frame}


  \begin{frame} \frametitle{Why ``computational economics''?}

    \begin{itemize}
      \item \textbf{Lucas critique}: The Lucas critique is the idea that if we attempt to build
        policies based on statistical relationships within data, then we don't know what will
        happen because people might change their behavior.
        \href{http://www.tomsargent.com/research/Critique_Consequence.pdf}{This paper} is an
        excellent read if you would like to know more
    \end{itemize}
  \end{frame}


  \begin{frame} \frametitle{Why ``computational economics''?}

    \begin{itemize}
      \item \textbf{Quantiative ``realism''}: Lots of economists build models but computational
        economists typically build a slight more complex model of the economy which cannot
        typically be solved by pen and paper -- This aim of this complexity is to provide a more
        precise quantification of what might occur under different economic policies.
    \end{itemize}
  \end{frame}

\end{document}
